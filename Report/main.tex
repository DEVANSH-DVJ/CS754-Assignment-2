\documentclass[fleqn, 11pt]{article}

\usepackage{amsmath}
\DeclareMathOperator*{\argmin}{arg\,min}
\usepackage{amssymb}
\usepackage{amsthm}
\usepackage{hyperref}
\usepackage{ulem}
\usepackage{enumitem}
\usepackage[left=0.75in, right=0.75in, bottom=0.75in]{geometry}
\usepackage{graphicx}

\newcommand{\bs}[1]{\boldsymbol{#1}}


\usepackage{array}
\usepackage{caption}
\usepackage{floatrow}
\usepackage{multirow}

\usepackage{chngcntr}
\counterwithin*{equation}{section}
\counterwithin*{equation}{subsection}

\usepackage{sectsty}
\sectionfont{\centering}

\usepackage[perpage]{footmisc}

\usepackage{fancyhdr}
\pagestyle{fancy}
\fancyhf{}
\lhead{190100036 \& 190100044}
\rhead{CS 754: Assignment 2}
\renewcommand{\footrulewidth}{1.0pt}
\cfoot{Page \thepage}

\setlength{\parindent}{0em}
\renewcommand{\arraystretch}{2}%

\title{Assignment 2: CS 754}
\author{ 
\begin{tabular}{|c|c|}
     \hline
     \textsf{Krushnakant Bhattad} & \textsf{Devansh Jain} \\
     \hline
     \textsf{190100036} & \textsf{190100044}\\
     \hline
\end{tabular}
}
\date{February 22, 2021}

\begin{document}

\maketitle
\tableofcontents
\thispagestyle{empty}
\setcounter{page}{0}

\newpage
\section*{Question 1}
\addcontentsline{toc}{section}{Question 1}
\setcounter{equation}{0}


\newpage
\section*{Question 2}
\addcontentsline{toc}{section}{Question 2}
\setcounter{equation}{0}


\newpage
\section*{Question 3}
\addcontentsline{toc}{section}{Question 3}
\setcounter{equation}{0}


\newpage
\section*{Question 4}
\addcontentsline{toc}{section}{Question 4}
\setcounter{equation}{0}

[In this answer, all norms are $\ell_2$ norms unless stated otherwise]

For integer $k = 1,2, \ldots ,n$, the restricted isometry constant(RIC) $\delta_k$ of 
a matrix $\bs{A}$ of size $m \times n$ is the smallest number such that for any
k-sparse vector $\bs{\theta}$, we have:
\begin{center}
    $ (1-\delta_k) || \bs{\theta} ||^2 \leq || \bs{A\theta} ||^2 
    \leq (1+\delta_k) || \bs{\theta} ||^2 $
\end{center}


\medskip

Let $s<t$.

Let $Q_s$ be the set of all $s$-sparse vectors, and $Q_t$ be the set of all 
$t$-sparse vectors. As $s<t$, any $s$-sparse vector is also a $t$-sparse vector.
Hence, we have $Q_s \subset Q_t$.

$\delta_s$ is the smallest number such that for all $ \bs{\theta} \in Q_s$, 
\begin{center}
    $ (1-\delta_s) || \bs{\theta} ||^2 \leq || \bs{A\theta} ||^2 
    \leq (1+\delta_s) || \bs{\theta} ||^2 $
\end{center}

And $\delta_t$ is the smallest number such that for all $ \bs{\theta} \in Q_t$, 
\begin{center}
    $ (1-\delta_t) || \bs{\theta} ||^2 \leq || \bs{A\theta} ||^2 
    \leq (1+\delta_t) || \bs{\theta} ||^2 $
\end{center}

We have to prove that $\delta_s \leq \delta_t$. 

On the contrary, let's assume $\delta_s > \delta_t$. 
As $Q_s \subset Q_t$, for all $ \bs{\theta} \in Q_s \subset Q_t$, we have 

\begin{center}
    $ (1-\delta_t) || \bs{\theta} ||^2 \leq || \bs{A\theta} ||^2 
    \leq (1+\delta_t) || \bs{\theta} ||^2 $
\end{center}

As $\delta_s$ was defined to be the smallest number satisfying above, 
it cannot be the case that $\delta_s > \delta_t$.


Hence, it must be the case that $\delta_s \leq \delta_t$.

\newpage
\section*{Question 5}
\setcounter{equation}{0}
\addcontentsline{toc}{section}{Question 5}


\newpage
\section*{Question 6}
\addcontentsline{toc}{section}{Question 6}
\setcounter{equation}{0}

Fix a $\lambda > 0$ and consider the LASSO Problem:

$J(\boldsymbol{x}) = || \bs{y}-\bs{\Phi} \bs{x} ||_2^2 + \lambda ||\bs{x}||_1 $

\smallskip

Suppose $\bs{r}$ minimizes $J(\cdot)$. 

\smallskip

We claim that, if $\epsilon = || \bs{y}-\bs{\Phi} \bs{r} ||_2$ 
then, $\bs{r}$ is also a solution to: 

P1: $\min_{\bs{x}} ||\bs{x}||_1$ s. t. $|| \bs{y}-\bs{\Phi} \bs{x} ||_2  \leq \epsilon $

\smallskip

To prove this, we do as follows:

Firstly, for all $\bs{x} \neq \bs{r}$, we will have:  
$J(\bs{x}) \geq J(\bs{r})$.

\medskip 

Consider all $\bs{x}$ for which $|| \bs{y}-\bs{\Phi} \bs{x} ||_2  \leq \epsilon$.

Due to both sides being positive, we also have- 
$|| \bs{y}-\bs{\Phi} \bs{x} ||_2^2  \leq \epsilon^2$

\medskip

which we can write as: 
$ 0 \leq \epsilon^2- || \bs{y}-\bs{\Phi} \bs{x} ||_2^2$

\medskip 

We do know that following holds true for all $\bs{x}$, so it does for our constrain too:

\medskip

$|| \bs{y}-\bs{\Phi} \bs{x} ||_2^2 + \lambda ||\bs{x}||_1 
\geq || \bs{y}-\bs{\Phi} \bs{r} ||_2^2 + \lambda ||\bs{r}||_1 
$


\medskip

$|| \bs{y}-\bs{\Phi} \bs{x} ||_2^2 + \lambda ||\bs{x}||_1 
\geq \epsilon^2 + \lambda ||\bs{r}||_1 
$

\medskip

Now using our constrain, we can write, 

\medskip

$  \lambda ( ||\bs{x}||_1 -   ||\bs{r}||_1  ) \geq 
\epsilon^2 - || \bs{y}-\bs{\Phi} \bs{x} ||_2^2 \geq 0
$

\medskip

As $\lambda>0$ by definition, we thus have $||\bs{x}||_1 -   ||\bs{r}||_1  > 0$ 
for all $\bs{x}$ for which $|| \bs{y}-\bs{\Phi} \bs{x} ||_2  \leq \epsilon$.

\medskip

That is, for all  $\bs{x}$ for which $|| \bs{y}-\bs{\Phi} \bs{x} ||_2  \leq \epsilon$, 
$  ||\bs{r}||_1 < ||\bs{x}||_1 $ 

\smallskip

Thus, $\bs{r}$ is also a solution to P1 for the value of $\epsilon$ stated above. 
$\qed$


\end{document}

